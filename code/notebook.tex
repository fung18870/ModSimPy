
% Default to the notebook output style

    


% Inherit from the specified cell style.




    
\documentclass[11pt]{article}

    
    
    \usepackage[T1]{fontenc}
    % Nicer default font (+ math font) than Computer Modern for most use cases
    \usepackage{mathpazo}

    % Basic figure setup, for now with no caption control since it's done
    % automatically by Pandoc (which extracts ![](path) syntax from Markdown).
    \usepackage{graphicx}
    % We will generate all images so they have a width \maxwidth. This means
    % that they will get their normal width if they fit onto the page, but
    % are scaled down if they would overflow the margins.
    \makeatletter
    \def\maxwidth{\ifdim\Gin@nat@width>\linewidth\linewidth
    \else\Gin@nat@width\fi}
    \makeatother
    \let\Oldincludegraphics\includegraphics
    % Set max figure width to be 80% of text width, for now hardcoded.
    \renewcommand{\includegraphics}[1]{\Oldincludegraphics[width=.8\maxwidth]{#1}}
    % Ensure that by default, figures have no caption (until we provide a
    % proper Figure object with a Caption API and a way to capture that
    % in the conversion process - todo).
    \usepackage{caption}
    \DeclareCaptionLabelFormat{nolabel}{}
    \captionsetup{labelformat=nolabel}

    \usepackage{adjustbox} % Used to constrain images to a maximum size 
    \usepackage{xcolor} % Allow colors to be defined
    \usepackage{enumerate} % Needed for markdown enumerations to work
    \usepackage{geometry} % Used to adjust the document margins
    \usepackage{amsmath} % Equations
    \usepackage{amssymb} % Equations
    \usepackage{textcomp} % defines textquotesingle
    % Hack from http://tex.stackexchange.com/a/47451/13684:
    \AtBeginDocument{%
        \def\PYZsq{\textquotesingle}% Upright quotes in Pygmentized code
    }
    \usepackage{upquote} % Upright quotes for verbatim code
    \usepackage{eurosym} % defines \euro
    \usepackage[mathletters]{ucs} % Extended unicode (utf-8) support
    \usepackage[utf8x]{inputenc} % Allow utf-8 characters in the tex document
    \usepackage{fancyvrb} % verbatim replacement that allows latex
    \usepackage{grffile} % extends the file name processing of package graphics 
                         % to support a larger range 
    % The hyperref package gives us a pdf with properly built
    % internal navigation ('pdf bookmarks' for the table of contents,
    % internal cross-reference links, web links for URLs, etc.)
    \usepackage{hyperref}
    \usepackage{longtable} % longtable support required by pandoc >1.10
    \usepackage{booktabs}  % table support for pandoc > 1.12.2
    \usepackage[inline]{enumitem} % IRkernel/repr support (it uses the enumerate* environment)
    \usepackage[normalem]{ulem} % ulem is needed to support strikethroughs (\sout)
                                % normalem makes italics be italics, not underlines
    

    
    
    % Colors for the hyperref package
    \definecolor{urlcolor}{rgb}{0,.145,.698}
    \definecolor{linkcolor}{rgb}{.71,0.21,0.01}
    \definecolor{citecolor}{rgb}{.12,.54,.11}

    % ANSI colors
    \definecolor{ansi-black}{HTML}{3E424D}
    \definecolor{ansi-black-intense}{HTML}{282C36}
    \definecolor{ansi-red}{HTML}{E75C58}
    \definecolor{ansi-red-intense}{HTML}{B22B31}
    \definecolor{ansi-green}{HTML}{00A250}
    \definecolor{ansi-green-intense}{HTML}{007427}
    \definecolor{ansi-yellow}{HTML}{DDB62B}
    \definecolor{ansi-yellow-intense}{HTML}{B27D12}
    \definecolor{ansi-blue}{HTML}{208FFB}
    \definecolor{ansi-blue-intense}{HTML}{0065CA}
    \definecolor{ansi-magenta}{HTML}{D160C4}
    \definecolor{ansi-magenta-intense}{HTML}{A03196}
    \definecolor{ansi-cyan}{HTML}{60C6C8}
    \definecolor{ansi-cyan-intense}{HTML}{258F8F}
    \definecolor{ansi-white}{HTML}{C5C1B4}
    \definecolor{ansi-white-intense}{HTML}{A1A6B2}

    % commands and environments needed by pandoc snippets
    % extracted from the output of `pandoc -s`
    \providecommand{\tightlist}{%
      \setlength{\itemsep}{0pt}\setlength{\parskip}{0pt}}
    \DefineVerbatimEnvironment{Highlighting}{Verbatim}{commandchars=\\\{\}}
    % Add ',fontsize=\small' for more characters per line
    \newenvironment{Shaded}{}{}
    \newcommand{\KeywordTok}[1]{\textcolor[rgb]{0.00,0.44,0.13}{\textbf{{#1}}}}
    \newcommand{\DataTypeTok}[1]{\textcolor[rgb]{0.56,0.13,0.00}{{#1}}}
    \newcommand{\DecValTok}[1]{\textcolor[rgb]{0.25,0.63,0.44}{{#1}}}
    \newcommand{\BaseNTok}[1]{\textcolor[rgb]{0.25,0.63,0.44}{{#1}}}
    \newcommand{\FloatTok}[1]{\textcolor[rgb]{0.25,0.63,0.44}{{#1}}}
    \newcommand{\CharTok}[1]{\textcolor[rgb]{0.25,0.44,0.63}{{#1}}}
    \newcommand{\StringTok}[1]{\textcolor[rgb]{0.25,0.44,0.63}{{#1}}}
    \newcommand{\CommentTok}[1]{\textcolor[rgb]{0.38,0.63,0.69}{\textit{{#1}}}}
    \newcommand{\OtherTok}[1]{\textcolor[rgb]{0.00,0.44,0.13}{{#1}}}
    \newcommand{\AlertTok}[1]{\textcolor[rgb]{1.00,0.00,0.00}{\textbf{{#1}}}}
    \newcommand{\FunctionTok}[1]{\textcolor[rgb]{0.02,0.16,0.49}{{#1}}}
    \newcommand{\RegionMarkerTok}[1]{{#1}}
    \newcommand{\ErrorTok}[1]{\textcolor[rgb]{1.00,0.00,0.00}{\textbf{{#1}}}}
    \newcommand{\NormalTok}[1]{{#1}}
    
    % Additional commands for more recent versions of Pandoc
    \newcommand{\ConstantTok}[1]{\textcolor[rgb]{0.53,0.00,0.00}{{#1}}}
    \newcommand{\SpecialCharTok}[1]{\textcolor[rgb]{0.25,0.44,0.63}{{#1}}}
    \newcommand{\VerbatimStringTok}[1]{\textcolor[rgb]{0.25,0.44,0.63}{{#1}}}
    \newcommand{\SpecialStringTok}[1]{\textcolor[rgb]{0.73,0.40,0.53}{{#1}}}
    \newcommand{\ImportTok}[1]{{#1}}
    \newcommand{\DocumentationTok}[1]{\textcolor[rgb]{0.73,0.13,0.13}{\textit{{#1}}}}
    \newcommand{\AnnotationTok}[1]{\textcolor[rgb]{0.38,0.63,0.69}{\textbf{\textit{{#1}}}}}
    \newcommand{\CommentVarTok}[1]{\textcolor[rgb]{0.38,0.63,0.69}{\textbf{\textit{{#1}}}}}
    \newcommand{\VariableTok}[1]{\textcolor[rgb]{0.10,0.09,0.49}{{#1}}}
    \newcommand{\ControlFlowTok}[1]{\textcolor[rgb]{0.00,0.44,0.13}{\textbf{{#1}}}}
    \newcommand{\OperatorTok}[1]{\textcolor[rgb]{0.40,0.40,0.40}{{#1}}}
    \newcommand{\BuiltInTok}[1]{{#1}}
    \newcommand{\ExtensionTok}[1]{{#1}}
    \newcommand{\PreprocessorTok}[1]{\textcolor[rgb]{0.74,0.48,0.00}{{#1}}}
    \newcommand{\AttributeTok}[1]{\textcolor[rgb]{0.49,0.56,0.16}{{#1}}}
    \newcommand{\InformationTok}[1]{\textcolor[rgb]{0.38,0.63,0.69}{\textbf{\textit{{#1}}}}}
    \newcommand{\WarningTok}[1]{\textcolor[rgb]{0.38,0.63,0.69}{\textbf{\textit{{#1}}}}}
    
    
    % Define a nice break command that doesn't care if a line doesn't already
    % exist.
    \def\br{\hspace*{\fill} \\* }
    % Math Jax compatability definitions
    \def\gt{>}
    \def\lt{<}
    % Document parameters
    \title{Coffee+Third+Pass}
    
    
    

    % Pygments definitions
    
\makeatletter
\def\PY@reset{\let\PY@it=\relax \let\PY@bf=\relax%
    \let\PY@ul=\relax \let\PY@tc=\relax%
    \let\PY@bc=\relax \let\PY@ff=\relax}
\def\PY@tok#1{\csname PY@tok@#1\endcsname}
\def\PY@toks#1+{\ifx\relax#1\empty\else%
    \PY@tok{#1}\expandafter\PY@toks\fi}
\def\PY@do#1{\PY@bc{\PY@tc{\PY@ul{%
    \PY@it{\PY@bf{\PY@ff{#1}}}}}}}
\def\PY#1#2{\PY@reset\PY@toks#1+\relax+\PY@do{#2}}

\expandafter\def\csname PY@tok@w\endcsname{\def\PY@tc##1{\textcolor[rgb]{0.73,0.73,0.73}{##1}}}
\expandafter\def\csname PY@tok@c\endcsname{\let\PY@it=\textit\def\PY@tc##1{\textcolor[rgb]{0.25,0.50,0.50}{##1}}}
\expandafter\def\csname PY@tok@cp\endcsname{\def\PY@tc##1{\textcolor[rgb]{0.74,0.48,0.00}{##1}}}
\expandafter\def\csname PY@tok@k\endcsname{\let\PY@bf=\textbf\def\PY@tc##1{\textcolor[rgb]{0.00,0.50,0.00}{##1}}}
\expandafter\def\csname PY@tok@kp\endcsname{\def\PY@tc##1{\textcolor[rgb]{0.00,0.50,0.00}{##1}}}
\expandafter\def\csname PY@tok@kt\endcsname{\def\PY@tc##1{\textcolor[rgb]{0.69,0.00,0.25}{##1}}}
\expandafter\def\csname PY@tok@o\endcsname{\def\PY@tc##1{\textcolor[rgb]{0.40,0.40,0.40}{##1}}}
\expandafter\def\csname PY@tok@ow\endcsname{\let\PY@bf=\textbf\def\PY@tc##1{\textcolor[rgb]{0.67,0.13,1.00}{##1}}}
\expandafter\def\csname PY@tok@nb\endcsname{\def\PY@tc##1{\textcolor[rgb]{0.00,0.50,0.00}{##1}}}
\expandafter\def\csname PY@tok@nf\endcsname{\def\PY@tc##1{\textcolor[rgb]{0.00,0.00,1.00}{##1}}}
\expandafter\def\csname PY@tok@nc\endcsname{\let\PY@bf=\textbf\def\PY@tc##1{\textcolor[rgb]{0.00,0.00,1.00}{##1}}}
\expandafter\def\csname PY@tok@nn\endcsname{\let\PY@bf=\textbf\def\PY@tc##1{\textcolor[rgb]{0.00,0.00,1.00}{##1}}}
\expandafter\def\csname PY@tok@ne\endcsname{\let\PY@bf=\textbf\def\PY@tc##1{\textcolor[rgb]{0.82,0.25,0.23}{##1}}}
\expandafter\def\csname PY@tok@nv\endcsname{\def\PY@tc##1{\textcolor[rgb]{0.10,0.09,0.49}{##1}}}
\expandafter\def\csname PY@tok@no\endcsname{\def\PY@tc##1{\textcolor[rgb]{0.53,0.00,0.00}{##1}}}
\expandafter\def\csname PY@tok@nl\endcsname{\def\PY@tc##1{\textcolor[rgb]{0.63,0.63,0.00}{##1}}}
\expandafter\def\csname PY@tok@ni\endcsname{\let\PY@bf=\textbf\def\PY@tc##1{\textcolor[rgb]{0.60,0.60,0.60}{##1}}}
\expandafter\def\csname PY@tok@na\endcsname{\def\PY@tc##1{\textcolor[rgb]{0.49,0.56,0.16}{##1}}}
\expandafter\def\csname PY@tok@nt\endcsname{\let\PY@bf=\textbf\def\PY@tc##1{\textcolor[rgb]{0.00,0.50,0.00}{##1}}}
\expandafter\def\csname PY@tok@nd\endcsname{\def\PY@tc##1{\textcolor[rgb]{0.67,0.13,1.00}{##1}}}
\expandafter\def\csname PY@tok@s\endcsname{\def\PY@tc##1{\textcolor[rgb]{0.73,0.13,0.13}{##1}}}
\expandafter\def\csname PY@tok@sd\endcsname{\let\PY@it=\textit\def\PY@tc##1{\textcolor[rgb]{0.73,0.13,0.13}{##1}}}
\expandafter\def\csname PY@tok@si\endcsname{\let\PY@bf=\textbf\def\PY@tc##1{\textcolor[rgb]{0.73,0.40,0.53}{##1}}}
\expandafter\def\csname PY@tok@se\endcsname{\let\PY@bf=\textbf\def\PY@tc##1{\textcolor[rgb]{0.73,0.40,0.13}{##1}}}
\expandafter\def\csname PY@tok@sr\endcsname{\def\PY@tc##1{\textcolor[rgb]{0.73,0.40,0.53}{##1}}}
\expandafter\def\csname PY@tok@ss\endcsname{\def\PY@tc##1{\textcolor[rgb]{0.10,0.09,0.49}{##1}}}
\expandafter\def\csname PY@tok@sx\endcsname{\def\PY@tc##1{\textcolor[rgb]{0.00,0.50,0.00}{##1}}}
\expandafter\def\csname PY@tok@m\endcsname{\def\PY@tc##1{\textcolor[rgb]{0.40,0.40,0.40}{##1}}}
\expandafter\def\csname PY@tok@gh\endcsname{\let\PY@bf=\textbf\def\PY@tc##1{\textcolor[rgb]{0.00,0.00,0.50}{##1}}}
\expandafter\def\csname PY@tok@gu\endcsname{\let\PY@bf=\textbf\def\PY@tc##1{\textcolor[rgb]{0.50,0.00,0.50}{##1}}}
\expandafter\def\csname PY@tok@gd\endcsname{\def\PY@tc##1{\textcolor[rgb]{0.63,0.00,0.00}{##1}}}
\expandafter\def\csname PY@tok@gi\endcsname{\def\PY@tc##1{\textcolor[rgb]{0.00,0.63,0.00}{##1}}}
\expandafter\def\csname PY@tok@gr\endcsname{\def\PY@tc##1{\textcolor[rgb]{1.00,0.00,0.00}{##1}}}
\expandafter\def\csname PY@tok@ge\endcsname{\let\PY@it=\textit}
\expandafter\def\csname PY@tok@gs\endcsname{\let\PY@bf=\textbf}
\expandafter\def\csname PY@tok@gp\endcsname{\let\PY@bf=\textbf\def\PY@tc##1{\textcolor[rgb]{0.00,0.00,0.50}{##1}}}
\expandafter\def\csname PY@tok@go\endcsname{\def\PY@tc##1{\textcolor[rgb]{0.53,0.53,0.53}{##1}}}
\expandafter\def\csname PY@tok@gt\endcsname{\def\PY@tc##1{\textcolor[rgb]{0.00,0.27,0.87}{##1}}}
\expandafter\def\csname PY@tok@err\endcsname{\def\PY@bc##1{\setlength{\fboxsep}{0pt}\fcolorbox[rgb]{1.00,0.00,0.00}{1,1,1}{\strut ##1}}}
\expandafter\def\csname PY@tok@kc\endcsname{\let\PY@bf=\textbf\def\PY@tc##1{\textcolor[rgb]{0.00,0.50,0.00}{##1}}}
\expandafter\def\csname PY@tok@kd\endcsname{\let\PY@bf=\textbf\def\PY@tc##1{\textcolor[rgb]{0.00,0.50,0.00}{##1}}}
\expandafter\def\csname PY@tok@kn\endcsname{\let\PY@bf=\textbf\def\PY@tc##1{\textcolor[rgb]{0.00,0.50,0.00}{##1}}}
\expandafter\def\csname PY@tok@kr\endcsname{\let\PY@bf=\textbf\def\PY@tc##1{\textcolor[rgb]{0.00,0.50,0.00}{##1}}}
\expandafter\def\csname PY@tok@bp\endcsname{\def\PY@tc##1{\textcolor[rgb]{0.00,0.50,0.00}{##1}}}
\expandafter\def\csname PY@tok@fm\endcsname{\def\PY@tc##1{\textcolor[rgb]{0.00,0.00,1.00}{##1}}}
\expandafter\def\csname PY@tok@vc\endcsname{\def\PY@tc##1{\textcolor[rgb]{0.10,0.09,0.49}{##1}}}
\expandafter\def\csname PY@tok@vg\endcsname{\def\PY@tc##1{\textcolor[rgb]{0.10,0.09,0.49}{##1}}}
\expandafter\def\csname PY@tok@vi\endcsname{\def\PY@tc##1{\textcolor[rgb]{0.10,0.09,0.49}{##1}}}
\expandafter\def\csname PY@tok@vm\endcsname{\def\PY@tc##1{\textcolor[rgb]{0.10,0.09,0.49}{##1}}}
\expandafter\def\csname PY@tok@sa\endcsname{\def\PY@tc##1{\textcolor[rgb]{0.73,0.13,0.13}{##1}}}
\expandafter\def\csname PY@tok@sb\endcsname{\def\PY@tc##1{\textcolor[rgb]{0.73,0.13,0.13}{##1}}}
\expandafter\def\csname PY@tok@sc\endcsname{\def\PY@tc##1{\textcolor[rgb]{0.73,0.13,0.13}{##1}}}
\expandafter\def\csname PY@tok@dl\endcsname{\def\PY@tc##1{\textcolor[rgb]{0.73,0.13,0.13}{##1}}}
\expandafter\def\csname PY@tok@s2\endcsname{\def\PY@tc##1{\textcolor[rgb]{0.73,0.13,0.13}{##1}}}
\expandafter\def\csname PY@tok@sh\endcsname{\def\PY@tc##1{\textcolor[rgb]{0.73,0.13,0.13}{##1}}}
\expandafter\def\csname PY@tok@s1\endcsname{\def\PY@tc##1{\textcolor[rgb]{0.73,0.13,0.13}{##1}}}
\expandafter\def\csname PY@tok@mb\endcsname{\def\PY@tc##1{\textcolor[rgb]{0.40,0.40,0.40}{##1}}}
\expandafter\def\csname PY@tok@mf\endcsname{\def\PY@tc##1{\textcolor[rgb]{0.40,0.40,0.40}{##1}}}
\expandafter\def\csname PY@tok@mh\endcsname{\def\PY@tc##1{\textcolor[rgb]{0.40,0.40,0.40}{##1}}}
\expandafter\def\csname PY@tok@mi\endcsname{\def\PY@tc##1{\textcolor[rgb]{0.40,0.40,0.40}{##1}}}
\expandafter\def\csname PY@tok@il\endcsname{\def\PY@tc##1{\textcolor[rgb]{0.40,0.40,0.40}{##1}}}
\expandafter\def\csname PY@tok@mo\endcsname{\def\PY@tc##1{\textcolor[rgb]{0.40,0.40,0.40}{##1}}}
\expandafter\def\csname PY@tok@ch\endcsname{\let\PY@it=\textit\def\PY@tc##1{\textcolor[rgb]{0.25,0.50,0.50}{##1}}}
\expandafter\def\csname PY@tok@cm\endcsname{\let\PY@it=\textit\def\PY@tc##1{\textcolor[rgb]{0.25,0.50,0.50}{##1}}}
\expandafter\def\csname PY@tok@cpf\endcsname{\let\PY@it=\textit\def\PY@tc##1{\textcolor[rgb]{0.25,0.50,0.50}{##1}}}
\expandafter\def\csname PY@tok@c1\endcsname{\let\PY@it=\textit\def\PY@tc##1{\textcolor[rgb]{0.25,0.50,0.50}{##1}}}
\expandafter\def\csname PY@tok@cs\endcsname{\let\PY@it=\textit\def\PY@tc##1{\textcolor[rgb]{0.25,0.50,0.50}{##1}}}

\def\PYZbs{\char`\\}
\def\PYZus{\char`\_}
\def\PYZob{\char`\{}
\def\PYZcb{\char`\}}
\def\PYZca{\char`\^}
\def\PYZam{\char`\&}
\def\PYZlt{\char`\<}
\def\PYZgt{\char`\>}
\def\PYZsh{\char`\#}
\def\PYZpc{\char`\%}
\def\PYZdl{\char`\$}
\def\PYZhy{\char`\-}
\def\PYZsq{\char`\'}
\def\PYZdq{\char`\"}
\def\PYZti{\char`\~}
% for compatibility with earlier versions
\def\PYZat{@}
\def\PYZlb{[}
\def\PYZrb{]}
\makeatother


    % Exact colors from NB
    \definecolor{incolor}{rgb}{0.0, 0.0, 0.5}
    \definecolor{outcolor}{rgb}{0.545, 0.0, 0.0}



    
    % Prevent overflowing lines due to hard-to-break entities
    \sloppy 
    % Setup hyperref package
    \hypersetup{
      breaklinks=true,  % so long urls are correctly broken across lines
      colorlinks=true,
      urlcolor=urlcolor,
      linkcolor=linkcolor,
      citecolor=citecolor,
      }
    % Slightly bigger margins than the latex defaults
    
    \geometry{verbose,tmargin=1in,bmargin=1in,lmargin=1in,rmargin=1in}
    
    

    \begin{document}
    
    
    \maketitle
    
    

    
    \hypertarget{olins-coffee-cooling-problem}{%
\section{Olin's Coffee Cooling
Problem}\label{olins-coffee-cooling-problem}}

\hypertarget{by-theo-johnson-and-mary-fung}{%
\paragraph{By: Theo Johnson and Mary
Fung}\label{by-theo-johnson-and-mary-fung}}

    \begin{Verbatim}[commandchars=\\\{\}]
{\color{incolor}In [{\color{incolor}1}]:} \PY{c+c1}{\PYZsh{} Configure Jupyter so figures appear in the notebook}
        \PY{o}{\PYZpc{}}\PY{k}{matplotlib} inline
        
        \PY{c+c1}{\PYZsh{} Configure Jupyter to display the assigned value after an assignment}
        \PY{o}{\PYZpc{}}\PY{k}{config} InteractiveShell.ast\PYZus{}node\PYZus{}interactivity=\PYZsq{}last\PYZus{}expr\PYZus{}or\PYZus{}assign\PYZsq{}
        
        \PY{c+c1}{\PYZsh{} import functions from the modsim.py module}
        \PY{k+kn}{from} \PY{n+nn}{modsim} \PY{k}{import} \PY{o}{*}
\end{Verbatim}


    \hypertarget{background}{%
\subsubsection{Background}\label{background}}

Picture this, it's a Thursday evening and ModSim is just wrapping up.
Before you leave, your teacher announces that the next class will start
in the Nord where the teaching team will spend an hour introducing the
next topic. By now, it's not uncommon for any of the first-year courses,
such as DesNat or ModSim, to meet in the Nord heading to the AC for time
in the studios. If only it weren't so cold in the Nord.

    Last time this happened, you were already tired and low on energy when
you got to the AC. Fortunately, you had made some coffee before heading
to class. Unfortunately, the colder air in the Nord and the frigid
weather outside have cooled your coffee more than you expected. This
time however, you're more prepared.

    \hypertarget{question}{%
\subsection{Question}\label{question}}

How warm should you make your coffee, knowing that you aren't going to
drink it until you get to the AC and that you'll be in the Nord for an
hour beforehand, and instead of making your coffee hotter, could you
change the \emph{k} value to get the same results?

    \hypertarget{method}{%
\subsection{Method}\label{method}}

Using the coffee cooling model in the Chapter 15 and 16 notebooks, we
created our initial model with a set initial coffee temperature of 90
degrees Celsius, we'll sweep the initial temperature once we have a
working model.

    \hypertarget{creating-a-model}{%
\subsubsection{Creating a Model}\label{creating-a-model}}

First things first, let's make our system.

Our system parameters include the initial temperature, the volume of
coffee (in mL), the \emph{k} value (for now let's assume 0.01), the
environment temperatures (here's where it gets complicated), the ending
time (in minutes), and the time step size (also in minutes).

Now, we also have to set the different temperatures for different times.
Assuming you leave the Dining Hall 3 minutes after making your coffee,
walk outside for 1 minute before arriving in the Nord, stay in the Nord
for an hour, walk outside for 3 minutes to the AC, and take 5 minutes to
settle down before drinking your coffee, the code looks like the
following.

    \begin{Verbatim}[commandchars=\\\{\}]
{\color{incolor}In [{\color{incolor}2}]:} \PY{n}{init} \PY{o}{=} \PY{n}{State}\PY{p}{(}\PY{n}{T}\PY{o}{=}\PY{l+m+mi}{90}\PY{p}{)} \PY{c+c1}{\PYZsh{}all temps in in C}
        \PY{n}{coffee} \PY{o}{=} \PY{n}{System}\PY{p}{(}\PY{n}{init}\PY{o}{=}\PY{n}{init}\PY{p}{,}
                        \PY{n}{volume}\PY{o}{=}\PY{l+m+mi}{200}\PY{p}{,} \PY{c+c1}{\PYZsh{}ml}
                        \PY{n}{k}\PY{o}{=}\PY{l+m+mf}{0.033}\PY{p}{,}
                        \PY{n}{t\PYZus{}end}\PY{o}{=}\PY{l+m+mi}{68}\PY{p}{,}
                        \PY{n}{dt}\PY{o}{=}\PY{l+m+mi}{1}\PY{p}{)}
        \PY{c+c1}{\PYZsh{}Set values vol, k, time, step size}
\end{Verbatim}


\begin{Verbatim}[commandchars=\\\{\}]
{\color{outcolor}Out[{\color{outcolor}2}]:} init      T    90
        dtype: int64
        volume                     200
        k                        0.033
        t\_end                       68
        dt                           1
        dtype: object
\end{Verbatim}
            
    \begin{Verbatim}[commandchars=\\\{\}]
{\color{incolor}In [{\color{incolor}3}]:} \PY{k}{for} \PY{n}{t} \PY{o+ow}{in} \PY{n+nb}{range}\PY{p}{(}\PY{l+m+mi}{1}\PY{p}{)}\PY{p}{:}
            \PY{l+s+sd}{\PYZdq{}\PYZdq{}\PYZdq{}Temperature in the Dining Hall was about 73 degrees F\PYZdq{}\PYZdq{}\PYZdq{}}
            \PY{n}{T\PYZus{}env} \PY{o}{=} \PY{l+m+mf}{22.78}
        
            
        \PY{k}{for} \PY{n}{t} \PY{o+ow}{in} \PY{n+nb}{range}\PY{p}{(}\PY{l+m+mi}{1}\PY{p}{,}\PY{l+m+mi}{2}\PY{p}{)}\PY{p}{:}
            \PY{l+s+sd}{\PYZdq{}\PYZdq{}\PYZdq{}Temperature outside in the end of October was 53 degrees F on average\PYZdq{}\PYZdq{}\PYZdq{}}
            \PY{n}{T\PYZus{}env} \PY{o}{=} \PY{l+m+mf}{11.67}
          
            
        \PY{k}{for} \PY{n}{t} \PY{o+ow}{in} \PY{n+nb}{range}\PY{p}{(}\PY{l+m+mi}{2}\PY{p}{,}\PY{l+m+mi}{62}\PY{p}{)}\PY{p}{:}
            \PY{l+s+sd}{\PYZdq{}\PYZdq{}\PYZdq{}Temperature in the Nord was about 68 degrees F\PYZdq{}\PYZdq{}\PYZdq{}}
            \PY{n}{T\PYZus{}env} \PY{o}{=} \PY{l+m+mi}{20}
         
            
        \PY{k}{for} \PY{n}{t} \PY{o+ow}{in} \PY{n+nb}{range}\PY{p}{(}\PY{l+m+mi}{62}\PY{p}{,}\PY{l+m+mi}{63}\PY{p}{)}\PY{p}{:}
            \PY{l+s+sd}{\PYZdq{}\PYZdq{}\PYZdq{}Temperature outside\PYZdq{}\PYZdq{}\PYZdq{}}
            \PY{n}{T\PYZus{}env} \PY{o}{=} \PY{l+m+mf}{11.67}
          
            
        \PY{k}{for} \PY{n}{t} \PY{o+ow}{in} \PY{n+nb}{range}\PY{p}{(}\PY{l+m+mi}{63}\PY{p}{,}\PY{l+m+mi}{68}\PY{p}{)}\PY{p}{:}
            \PY{l+s+sd}{\PYZdq{}\PYZdq{}\PYZdq{}Temperature in the AC was about 70 degrees F\PYZdq{}\PYZdq{}\PYZdq{}}
            \PY{n}{T\PYZus{}env} \PY{o}{=} \PY{l+m+mf}{21.11}
            
        \PY{c+c1}{\PYZsh{}set external temps in each time frame, timeframes in order:}
        \PY{c+c1}{\PYZsh{} DH, outside, Nord, outside, AC}
\end{Verbatim}


    Newton's Law of Cooling represents our differential equation of our
stock-and-flow diagram, we'll store that in our update function.

    \begin{Verbatim}[commandchars=\\\{\}]
{\color{incolor}In [{\color{incolor}4}]:} \PY{k}{def} \PY{n+nf}{update\PYZus{}func}\PY{p}{(}\PY{n}{state}\PY{p}{,} \PY{n}{t}\PY{p}{,} \PY{n}{system}\PY{p}{)}\PY{p}{:}
            \PY{l+s+sd}{\PYZdq{}\PYZdq{}\PYZdq{}Update the thermal transfer model.}
        \PY{l+s+sd}{    }
        \PY{l+s+sd}{    state: State (temperature in degrees Celsius)}
        \PY{l+s+sd}{    t: time in minutes}
        \PY{l+s+sd}{    system: System object}
        \PY{l+s+sd}{    }
        \PY{l+s+sd}{    returns: State (temperature)}
        \PY{l+s+sd}{    \PYZdq{}\PYZdq{}\PYZdq{}}
            \PY{n}{unpack}\PY{p}{(}\PY{n}{system}\PY{p}{)}
            
            \PY{n}{T} \PY{o}{=} \PY{n}{state}\PY{o}{.}\PY{n}{T}
            \PY{n}{T} \PY{o}{+}\PY{o}{=} \PY{o}{\PYZhy{}}\PY{n}{k} \PY{o}{*} \PY{p}{(}\PY{n}{T} \PY{o}{\PYZhy{}} \PY{n}{T\PYZus{}env}\PY{p}{)} \PY{o}{*} \PY{n}{dt}
            \PY{c+c1}{\PYZsh{}Gets the temperature from State}
            \PY{c+c1}{\PYZsh{}Adds the change in temperature}
            \PY{c+c1}{\PYZsh{}Returns the new temperature to State}
            \PY{k}{return} \PY{n}{State}\PY{p}{(}\PY{n}{T}\PY{o}{=}\PY{n}{T}\PY{p}{)}
        
        \PY{c+c1}{\PYZsh{}Differential equation from Newton\PYZsq{}s law of Cooling}
        \PY{c+c1}{\PYZsh{}T=init temp, k=set value from cup, T\PYZus{}env = external temp, dt= step size}
\end{Verbatim}


    Let's see if it works:

    \begin{Verbatim}[commandchars=\\\{\}]
{\color{incolor}In [{\color{incolor}5}]:} \PY{n}{update\PYZus{}func}\PY{p}{(}\PY{n}{init}\PY{p}{,} \PY{l+m+mi}{0}\PY{p}{,} \PY{n}{coffee}\PY{p}{)}
        \PY{c+c1}{\PYZsh{}first step to test function}
\end{Verbatim}


\begin{Verbatim}[commandchars=\\\{\}]
{\color{outcolor}Out[{\color{outcolor}5}]:} T    87.72663
        dtype: float64
\end{Verbatim}
            
    Now our run simulation function:

    \begin{Verbatim}[commandchars=\\\{\}]
{\color{incolor}In [{\color{incolor}6}]:} \PY{k}{def} \PY{n+nf}{run\PYZus{}simulation}\PY{p}{(}\PY{n}{system}\PY{p}{,} \PY{n}{update\PYZus{}func}\PY{p}{)}\PY{p}{:}
            \PY{l+s+sd}{\PYZdq{}\PYZdq{}\PYZdq{}Runs a simulation of the system.}
        \PY{l+s+sd}{    }
        \PY{l+s+sd}{    Add a TimeFrame to the System: results}
        \PY{l+s+sd}{    }
        \PY{l+s+sd}{    system: System object}
        \PY{l+s+sd}{    update\PYZus{}func: function that updates state}
        \PY{l+s+sd}{    \PYZdq{}\PYZdq{}\PYZdq{}}
            \PY{n}{unpack}\PY{p}{(}\PY{n}{system}\PY{p}{)}
            
            \PY{n}{frame} \PY{o}{=} \PY{n}{TimeFrame}\PY{p}{(}\PY{n}{columns}\PY{o}{=}\PY{n}{init}\PY{o}{.}\PY{n}{index}\PY{p}{)}
            \PY{n}{frame}\PY{o}{.}\PY{n}{row}\PY{p}{[}\PY{l+m+mi}{0}\PY{p}{]} \PY{o}{=} \PY{n}{init}
            \PY{n}{ts} \PY{o}{=} \PY{n}{linrange}\PY{p}{(}\PY{l+m+mi}{0}\PY{p}{,} \PY{n}{t\PYZus{}end}\PY{p}{,} \PY{n}{dt}\PY{p}{)}
            
            \PY{k}{for} \PY{n}{t} \PY{o+ow}{in} \PY{n}{ts}\PY{p}{:}
                \PY{n}{frame}\PY{o}{.}\PY{n}{row}\PY{p}{[}\PY{n}{t}\PY{o}{+}\PY{n}{dt}\PY{p}{]} \PY{o}{=} \PY{n}{update\PYZus{}func}\PY{p}{(}\PY{n}{frame}\PY{o}{.}\PY{n}{row}\PY{p}{[}\PY{n}{t}\PY{p}{]}\PY{p}{,} \PY{n}{t}\PY{p}{,} \PY{n}{system}\PY{p}{)}
                
            \PY{c+c1}{\PYZsh{} store the final temperature in T\PYZus{}final}
            \PY{n}{system}\PY{o}{.}\PY{n}{T\PYZus{}final} \PY{o}{=} \PY{n}{get\PYZus{}last\PYZus{}value}\PY{p}{(}\PY{n}{frame}\PY{o}{.}\PY{n}{T}\PY{p}{)}
            
            \PY{k}{return} \PY{n}{frame}
\end{Verbatim}


    To put it all together:

    \begin{Verbatim}[commandchars=\\\{\}]
{\color{incolor}In [{\color{incolor}7}]:} \PY{n}{results} \PY{o}{=} \PY{n}{run\PYZus{}simulation}\PY{p}{(}\PY{n}{coffee}\PY{p}{,} \PY{n}{update\PYZus{}func}\PY{p}{)}
\end{Verbatim}


\begin{Verbatim}[commandchars=\\\{\}]
{\color{outcolor}Out[{\color{outcolor}7}]:}           T
        0        90
        1   87.7266
        2   85.5283
        3   83.4025
        4   81.3468
        5    79.359
        6   77.4368
        7    75.578
        8   73.7806
        9   72.0424
        10  70.3617
        11  68.7364
        12  67.1647
        13  65.6449
        14  64.1752
        15  62.7541
        16  61.3798
        17  60.0509
        18  58.7659
        19  57.5232
        20  56.3216
        21  55.1596
        22   54.036
        23  52.9494
        24  51.8987
        25  50.8827
        26  49.9002
        27  48.9501
        28  48.0314
        29   47.143
        ..      {\ldots}
        39  39.7218
        40  39.1076
        41  38.5137
        42  37.9394
        43   37.384
        44   36.847
        45  36.3276
        46  35.8255
        47  35.3399
        48  34.8703
        49  34.4162
        50  33.9771
        51  33.5525
        52  33.1419
        53  32.7448
        54  32.3609
        55  31.9896
        56  31.6306
        57  31.2834
        58  30.9477
        59   30.623
        60  30.3091
        61  30.0055
        62   29.712
        63  29.4281
        64  29.1536
        65  28.8882
        66  28.6315
        67  28.3833
        68  28.1433
        
        [69 rows x 1 columns]
\end{Verbatim}
            
    As a graph, it's easier to see the change in temperature that occurs.

    \begin{Verbatim}[commandchars=\\\{\}]
{\color{incolor}In [{\color{incolor}8}]:} \PY{n}{plot}\PY{p}{(}\PY{n}{results}\PY{o}{.}\PY{n}{T}\PY{p}{,} \PY{n}{label}\PY{o}{=}\PY{l+s+s1}{\PYZsq{}}\PY{l+s+s1}{coffee}\PY{l+s+s1}{\PYZsq{}}\PY{p}{)}
        \PY{n}{decorate}\PY{p}{(}\PY{n}{xlabel}\PY{o}{=}\PY{l+s+s1}{\PYZsq{}}\PY{l+s+s1}{Time (minutes)}\PY{l+s+s1}{\PYZsq{}}\PY{p}{,}
                 \PY{n}{ylabel}\PY{o}{=}\PY{l+s+s1}{\PYZsq{}}\PY{l+s+s1}{Temperature (C)}\PY{l+s+s1}{\PYZsq{}}\PY{p}{)}
        \PY{c+c1}{\PYZsh{}results of k value= .033 (Olin coffee cup)}
        \PY{c+c1}{\PYZsh{}and T=70 (reccomended serving heat of coffee)}
\end{Verbatim}


    \begin{center}
    \adjustimage{max size={0.9\linewidth}{0.9\paperheight}}{output_18_0.png}
    \end{center}
    { \hspace*{\fill} \\}
    
    From here, we can collect the final temperature of the coffee.

    \begin{Verbatim}[commandchars=\\\{\}]
{\color{incolor}In [{\color{incolor}9}]:} \PY{n}{coffee}\PY{o}{.}\PY{n}{T\PYZus{}final}
        \PY{c+c1}{\PYZsh{}final temp of coffee with Olin values}
\end{Verbatim}


\begin{Verbatim}[commandchars=\\\{\}]
{\color{outcolor}Out[{\color{outcolor}9}]:} 28.143255951829225
\end{Verbatim}
            
    \hypertarget{results-1}{%
\subsection{Results 1}\label{results-1}}

This is the skeleton of our model. As shown in the graph, the
temperature of the coffee continuously decreases, nearing the
temperature of the environment. It should be noted that no largely
noticeable difference in the temperature drop based on the change in
location can be seen. This is likely due to the fact that the
temperature of the coffee is much greater than the surrounding
environment, any change in rate is hard to see compared to the overall
decrease in temperature.

However, as it stands, the final temperature of our coffee, is 26
degrees Celsius (78.8 degrees Fahrenheit). This is rather cold, so we
will be attempting to raise the final temperature by changing both
\texttt{T\_init} and \texttt{k}.

    \hypertarget{cleaning-up-the-code}{%
\subsubsection{Cleaning up the code}\label{cleaning-up-the-code}}

Before we can move on, we should clean up our code. This will make it
easier moving forward. To do this, we used a make system function to
store our parameters.

    \begin{Verbatim}[commandchars=\\\{\}]
{\color{incolor}In [{\color{incolor}10}]:} \PY{k}{def} \PY{n+nf}{make\PYZus{}system}\PY{p}{(}\PY{n}{T\PYZus{}init}\PY{p}{,} \PY{n}{k}\PY{p}{,} \PY{n}{volume}\PY{p}{,} \PY{n}{t\PYZus{}end}\PY{p}{)}\PY{p}{:}
             \PY{l+s+sd}{\PYZdq{}\PYZdq{}\PYZdq{}Makes a System object with the given parameters.}
         
         \PY{l+s+sd}{    T\PYZus{}init: initial temperature in degC}
         \PY{l+s+sd}{    r: heat transfer rate, in 1/min}
         \PY{l+s+sd}{    volume: volume of liquid in mL}
         \PY{l+s+sd}{    t\PYZus{}end: end time of simulation}
         \PY{l+s+sd}{    }
         \PY{l+s+sd}{    returns: System object}
         \PY{l+s+sd}{    \PYZdq{}\PYZdq{}\PYZdq{}}
             \PY{n}{init} \PY{o}{=} \PY{n}{State}\PY{p}{(}\PY{n}{T}\PY{o}{=}\PY{n}{T\PYZus{}init}\PY{p}{)}
             
             \PY{c+c1}{\PYZsh{} T\PYZus{}final is used to store the final temperature.}
             \PY{c+c1}{\PYZsh{} Before the simulation runs, T\PYZus{}final = T\PYZus{}init}
             \PY{n}{T\PYZus{}final} \PY{o}{=} \PY{n}{T\PYZus{}init}
             
             \PY{c+c1}{\PYZsh{}T\PYZus{}env changes bsed on the location which corresponds to time}
             \PY{k}{for} \PY{n}{t} \PY{o+ow}{in} \PY{n+nb}{range}\PY{p}{(}\PY{l+m+mi}{1}\PY{p}{)}\PY{p}{:}
                 \PY{n}{T\PYZus{}env} \PY{o}{=} \PY{l+m+mf}{22.78}
         
         
             \PY{k}{for} \PY{n}{t} \PY{o+ow}{in} \PY{n+nb}{range}\PY{p}{(}\PY{l+m+mi}{1}\PY{p}{,}\PY{l+m+mi}{2}\PY{p}{)}\PY{p}{:}
                 \PY{n}{T\PYZus{}env} \PY{o}{=} \PY{l+m+mf}{11.67}
         
         
             \PY{k}{for} \PY{n}{t} \PY{o+ow}{in} \PY{n+nb}{range}\PY{p}{(}\PY{l+m+mi}{2}\PY{p}{,}\PY{l+m+mi}{62}\PY{p}{)}\PY{p}{:}
                 \PY{n}{T\PYZus{}env} \PY{o}{=} \PY{l+m+mi}{20}
         
         
             \PY{k}{for} \PY{n}{t} \PY{o+ow}{in} \PY{n+nb}{range}\PY{p}{(}\PY{l+m+mi}{62}\PY{p}{,}\PY{l+m+mi}{63}\PY{p}{)}\PY{p}{:}
                 \PY{n}{T\PYZus{}env} \PY{o}{=} \PY{l+m+mf}{11.67}
         
         
             \PY{k}{for} \PY{n}{t} \PY{o+ow}{in} \PY{n+nb}{range}\PY{p}{(}\PY{l+m+mi}{63}\PY{p}{,}\PY{l+m+mi}{68}\PY{p}{)}\PY{p}{:}
                 \PY{n}{T\PYZus{}env} \PY{o}{=} \PY{l+m+mf}{21.11}
             \PY{n}{dt} \PY{o}{=} \PY{l+m+mi}{1}
                         
             \PY{k}{return} \PY{n}{System}\PY{p}{(}\PY{n+nb}{locals}\PY{p}{(}\PY{p}{)}\PY{p}{)}
\end{Verbatim}


    Running the new code:

    \begin{Verbatim}[commandchars=\\\{\}]
{\color{incolor}In [{\color{incolor}11}]:} \PY{n}{coffee} \PY{o}{=} \PY{n}{make\PYZus{}system}\PY{p}{(}\PY{n}{T\PYZus{}init}\PY{o}{=}\PY{l+m+mi}{70}\PY{p}{,} \PY{n}{k}\PY{o}{=}\PY{l+m+mf}{0.033}\PY{p}{,} \PY{n}{volume}\PY{o}{=}\PY{l+m+mi}{200}\PY{p}{,} \PY{n}{t\PYZus{}end}\PY{o}{=}\PY{l+m+mi}{68}\PY{p}{)}
         \PY{n}{results} \PY{o}{=} \PY{n}{run\PYZus{}simulation}\PY{p}{(}\PY{n}{coffee}\PY{p}{,} \PY{n}{update\PYZus{}func}\PY{p}{)}
         \PY{n}{coffee}\PY{o}{.}\PY{n}{T\PYZus{}final}
\end{Verbatim}


\begin{Verbatim}[commandchars=\\\{\}]
{\color{outcolor}Out[{\color{outcolor}11}]:} 26.101375867105983
\end{Verbatim}
            
    And we can make it easier to get the final temperature by adding another
function. This will be necessary when we get to sweeping our parameters.

    \begin{Verbatim}[commandchars=\\\{\}]
{\color{incolor}In [{\color{incolor}12}]:} \PY{k}{def} \PY{n+nf}{coffee\PYZus{}final}\PY{p}{(}\PY{n}{results}\PY{p}{)}\PY{p}{:}
          
             \PY{k}{return} \PY{n}{get\PYZus{}last\PYZus{}value}\PY{p}{(}\PY{n}{results}\PY{o}{.}\PY{n}{T}\PY{p}{)}
\end{Verbatim}


    \hypertarget{sweeping-parameters}{%
\subsubsection{Sweeping parameters}\label{sweeping-parameters}}

Now that we have our model, let's answer our question by sweeping for
\texttt{T\_init} and \texttt{k}.

    \hypertarget{sweeping-t_init}{%
\paragraph{Sweeping T\_init}\label{sweeping-t_init}}

To sweep our initial temperature, we'll loop an array into a function
and return a SweepSeries object.

    \begin{Verbatim}[commandchars=\\\{\}]
{\color{incolor}In [{\color{incolor}13}]:} \PY{n}{T\PYZus{}array} \PY{o}{=} \PY{n}{linspace}\PY{p}{(}\PY{l+m+mi}{60}\PY{p}{,} \PY{l+m+mi}{100}\PY{p}{,} \PY{l+m+mi}{9}\PY{p}{)}
\end{Verbatim}


\begin{Verbatim}[commandchars=\\\{\}]
{\color{outcolor}Out[{\color{outcolor}13}]:} array([ 60.,  65.,  70.,  75.,  80.,  85.,  90.,  95., 100.])
\end{Verbatim}
            
    \begin{Verbatim}[commandchars=\\\{\}]
{\color{incolor}In [{\color{incolor}14}]:} \PY{k}{def} \PY{n+nf}{sweep\PYZus{}T}\PY{p}{(}\PY{n}{T\PYZus{}array}\PY{p}{,} \PY{n}{k}\PY{p}{)}\PY{p}{:}
             \PY{l+s+sd}{\PYZdq{}\PYZdq{}\PYZdq{}Sweep a range of values for T\PYZus{}init.}
         \PY{l+s+sd}{    }
         \PY{l+s+sd}{    T\PYZus{}array: array of T\PYZus{}init values}
         \PY{l+s+sd}{    r: rate of heat transfer}
         \PY{l+s+sd}{    }
         \PY{l+s+sd}{    returns: SweepSeries that maps from beta to total infected}
         \PY{l+s+sd}{    \PYZdq{}\PYZdq{}\PYZdq{}}
             \PY{n}{sweep} \PY{o}{=} \PY{n}{SweepSeries}\PY{p}{(}\PY{p}{)}
             \PY{k}{for} \PY{n}{T\PYZus{}init} \PY{o+ow}{in} \PY{n}{T\PYZus{}array}\PY{p}{:}
                 \PY{n}{coffee} \PY{o}{=} \PY{n}{make\PYZus{}system}\PY{p}{(}\PY{n}{T\PYZus{}init}\PY{p}{,} \PY{n}{k}\PY{p}{,} \PY{n}{volume}\PY{p}{,} \PY{n}{t\PYZus{}end}\PY{p}{)}
                 \PY{n}{results} \PY{o}{=} \PY{n}{run\PYZus{}simulation}\PY{p}{(}\PY{n}{coffee}\PY{p}{,} \PY{n}{update\PYZus{}func}\PY{p}{)}
                 \PY{n}{sweep}\PY{p}{[}\PY{n}{coffee}\PY{o}{.}\PY{n}{T\PYZus{}init}\PY{p}{]} \PY{o}{=} \PY{n}{coffee\PYZus{}final}\PY{p}{(}\PY{n}{results}\PY{p}{)}
             \PY{k}{return} \PY{n}{sweep}
\end{Verbatim}


    To test our results we can run our function and plot the results.

    \begin{Verbatim}[commandchars=\\\{\}]
{\color{incolor}In [{\color{incolor}15}]:} \PY{n}{init\PYZus{}Temp} \PY{o}{=} \PY{n}{sweep\PYZus{}T}\PY{p}{(}\PY{n}{T\PYZus{}array}\PY{p}{,} \PY{n}{k}\PY{p}{)}
\end{Verbatim}


\begin{Verbatim}[commandchars=\\\{\}]
{\color{outcolor}Out[{\color{outcolor}15}]:} 60     25.080436
         65     25.590906
         70     26.101376
         75     26.611846
         80     27.122316
         85     27.632786
         90     28.143256
         95     28.653726
         100    29.164196
         dtype: float64
\end{Verbatim}
            
    \begin{Verbatim}[commandchars=\\\{\}]
{\color{incolor}In [{\color{incolor}16}]:} \PY{n}{label} \PY{o}{=} \PY{l+s+s1}{\PYZsq{}}\PY{l+s+s1}{k = }\PY{l+s+s1}{\PYZsq{}} \PY{o}{+} \PY{n+nb}{str}\PY{p}{(}\PY{n}{k}\PY{p}{)}
         \PY{n}{plot}\PY{p}{(}\PY{n}{init\PYZus{}Temp}\PY{p}{,} \PY{n}{label}\PY{o}{=}\PY{n}{label}\PY{p}{)}
         
         \PY{n}{decorate}\PY{p}{(}\PY{n}{xlabel}\PY{o}{=}\PY{l+s+s1}{\PYZsq{}}\PY{l+s+s1}{Initial Temperature (degrees Celsius)}\PY{l+s+s1}{\PYZsq{}}\PY{p}{,}
                  \PY{n}{ylabel}\PY{o}{=}\PY{l+s+s1}{\PYZsq{}}\PY{l+s+s1}{Final Temperature}\PY{l+s+s1}{\PYZsq{}}\PY{p}{)}
\end{Verbatim}


    \begin{center}
    \adjustimage{max size={0.9\linewidth}{0.9\paperheight}}{output_34_0.png}
    \end{center}
    { \hspace*{\fill} \\}
    
    \hypertarget{results-2}{%
\subsection{Results 2}\label{results-2}}

As you can see, the higher the initial temperature, the higher the final
temperature of the coffee. With a constant \texttt{k} value of 0.033,
the highest possible initial temperature (since water boils at 100
degrees Celsius) still won't produce a high enough final temperature.
Additionally, it is unfeasible to carry around extremely hot coffee,
since the chance of spilling and burning yourself are high. Therefore,
we should look at other ways to produce higher final temperatures.

    \hypertarget{sweeping-t_init-and-k}{%
\paragraph{Sweeping T\_init and k}\label{sweeping-t_init-and-k}}

From here, we can sweep \emph{k} as well and use a SweepFrame to store
our results. By changing the rate of heat transfer, \texttt{k}, we can
also attempt to end our simulation with a better final temperature.

    \begin{Verbatim}[commandchars=\\\{\}]
{\color{incolor}In [{\color{incolor}17}]:} \PY{k}{def} \PY{n+nf}{sweep\PYZus{}parameters}\PY{p}{(}\PY{n}{T\PYZus{}array}\PY{p}{,} \PY{n}{k\PYZus{}array}\PY{p}{)}\PY{p}{:}
             \PY{l+s+sd}{\PYZdq{}\PYZdq{}\PYZdq{}Sweep a range of values for T\PYZus{}init and k.}
         \PY{l+s+sd}{    }
         \PY{l+s+sd}{    T\PYZus{}array: array of initial temperatures}
         \PY{l+s+sd}{    k\PYZus{}array: array of heat transfer rates}
         \PY{l+s+sd}{    }
         \PY{l+s+sd}{    returns: SweepFrame with one row for each T\PYZus{}init}
         \PY{l+s+sd}{             and one column for each k}
         \PY{l+s+sd}{    \PYZdq{}\PYZdq{}\PYZdq{}}
             \PY{n}{frame} \PY{o}{=} \PY{n}{SweepFrame}\PY{p}{(}\PY{n}{columns}\PY{o}{=}\PY{n}{k\PYZus{}array}\PY{p}{)}
             \PY{k}{for} \PY{n}{k} \PY{o+ow}{in} \PY{n}{k\PYZus{}array}\PY{p}{:}
                 \PY{n}{frame}\PY{p}{[}\PY{n}{k}\PY{p}{]} \PY{o}{=} \PY{n}{sweep\PYZus{}T}\PY{p}{(}\PY{n}{T\PYZus{}array}\PY{p}{,} \PY{n}{k}\PY{p}{)}
             \PY{k}{return} \PY{n}{frame}
\end{Verbatim}


    And the results:

    \begin{Verbatim}[commandchars=\\\{\}]
{\color{incolor}In [{\color{incolor}18}]:} \PY{n}{T\PYZus{}array} \PY{o}{=} \PY{n}{linspace}\PY{p}{(}\PY{l+m+mi}{60}\PY{p}{,} \PY{l+m+mi}{100}\PY{p}{,} \PY{l+m+mi}{9}\PY{p}{)}
         \PY{n}{k\PYZus{}array} \PY{o}{=} \PY{n}{linspace}\PY{p}{(}\PY{l+m+mi}{0}\PY{p}{,} \PY{o}{.}\PY{l+m+mi}{05}\PY{p}{,} \PY{l+m+mi}{6}\PY{p}{)}
         \PY{n}{frame} \PY{o}{=} \PY{n}{sweep\PYZus{}parameters}\PY{p}{(}\PY{n}{T\PYZus{}array}\PY{p}{,} \PY{n}{k\PYZus{}array}\PY{p}{)}
         \PY{n}{frame}\PY{o}{.}\PY{n}{head}\PY{p}{(}\PY{p}{)}
         \PY{c+c1}{\PYZsh{}two parameter sweep}
\end{Verbatim}


\begin{Verbatim}[commandchars=\\\{\}]
{\color{outcolor}Out[{\color{outcolor}18}]:}     0.00       0.01       0.02       0.03       0.04       0.05
         60  60.0  40.745012  30.954895  26.011352  23.532600  22.298620
         65  65.0  43.269442  32.220631  26.641508  23.844069  22.451438
         70  70.0  45.793871  33.486367  27.271663  24.155537  22.604257
         75  75.0  48.318301  34.752103  27.901819  24.467005  22.757075
         80  80.0  50.842730  36.017839  28.531975  24.778474  22.909893
\end{Verbatim}
            
    \begin{Verbatim}[commandchars=\\\{\}]
{\color{incolor}In [{\color{incolor}19}]:} \PY{n}{frame}\PY{o}{.}\PY{n}{tail}\PY{p}{(}\PY{p}{)}
         \PY{c+c1}{\PYZsh{}two parameter sweep}
\end{Verbatim}


\begin{Verbatim}[commandchars=\\\{\}]
{\color{outcolor}Out[{\color{outcolor}19}]:}       0.00       0.01       0.02       0.03       0.04       0.05
         80    80.0  50.842730  36.017839  28.531975  24.778474  22.909893
         85    85.0  53.367159  37.283575  29.162131  25.089942  23.062711
         90    90.0  55.891589  38.549311  29.792287  25.401410  23.215530
         95    95.0  58.416018  39.815047  30.422442  25.712879  23.368348
         100  100.0  60.940448  41.080783  31.052598  26.024347  23.521166
\end{Verbatim}
            
    And to plot the results:

    \begin{Verbatim}[commandchars=\\\{\}]
{\color{incolor}In [{\color{incolor}20}]:} \PY{k}{for} \PY{n}{k} \PY{o+ow}{in} \PY{n}{k\PYZus{}array}\PY{p}{:}
             \PY{n}{label} \PY{o}{=} \PY{l+s+s1}{\PYZsq{}}\PY{l+s+s1}{k = }\PY{l+s+s1}{\PYZsq{}} \PY{o}{+} \PY{n+nb}{str}\PY{p}{(}\PY{n}{k}\PY{p}{)}
             \PY{n}{plot}\PY{p}{(}\PY{n}{frame}\PY{p}{[}\PY{n}{k}\PY{p}{]}\PY{p}{,} \PY{n}{label}\PY{o}{=}\PY{n}{label}\PY{p}{)}
             
         \PY{n}{decorate}\PY{p}{(}\PY{n}{xlabel}\PY{o}{=}\PY{l+s+s1}{\PYZsq{}}\PY{l+s+s1}{Starting Temp. (C)}\PY{l+s+s1}{\PYZsq{}}\PY{p}{,}
                  \PY{n}{ylabel}\PY{o}{=}\PY{l+s+s1}{\PYZsq{}}\PY{l+s+s1}{Final Temp (C)}\PY{l+s+s1}{\PYZsq{}}\PY{p}{,}
                  \PY{n}{loc}\PY{o}{=}\PY{l+m+mi}{1}\PY{p}{)}
         \PY{c+c1}{\PYZsh{}plot of two parameter sweep}
\end{Verbatim}


    \begin{center}
    \adjustimage{max size={0.9\linewidth}{0.9\paperheight}}{output_42_0.png}
    \end{center}
    { \hspace*{\fill} \\}
    
    \hypertarget{results-3}{%
\subsection{Results 3}\label{results-3}}

Changing the rate of heat transfer has a huge impact on the final
temperature, especially when used alongside different starting
temperatures. Based on these results, we can see that lower \texttt{k}
values have better effects on the final temperature of the coffee.
Looking at the results, \texttt{k} values of 0.02 and lower can, at
different initial temperatures result in final coffee temperatures that
are more suitable for drinking.

    \hypertarget{interpretation}{%
\subsection{Interpretation}\label{interpretation}}

The temperature range that is suggested to be best for drinking coffee
lies between 40 and 60 degrees Celsius (around 100 to 140 F), depending
on each person's personal preference. The recommended serving
temperature for coffee is between 70 and 85 degrees Celsius (around 160
to 185 F).

To meet this requirement with a coffee cup with a k value of 0.03, which
was our original k value, the initial temperature has to be almost 100
degrees Celsius which is the boiling temperature of water. Not only is
this unfeasible, but coffee that hot would be dangerous to carry around
since it would be very easy to burn oneself.

Comparatively, having a cup with a k value of 0.01 allows for an initial
temperature of 80 C or higher and produces a final temperature between
50 C and 60 C.

    No model is perfect, and ours certainly had a fair amount of failings.

First of all, we had hoped to see a change in the cooling of the coffee
based on the change in location. While a difference does exist, it is
barely noticeable and all we can see is a general decrease in
temperature over time. This short falling was able to be ignored in the
larger scheme of things, since our question depended more on the
eventual final temperature of the coffee after a significant decrease in
temperature had occurred.

Another alteration we had to make to the model occurred when we began
sweeping parameters. Originally, we had intended to only sweep for the
initial temperature. This proved to be too simple of a model and didn't
allow for the question to be answered based on the k value we originally
had as a constant. To solve this, we added a second parameter sweep.

Throughout the project, iteration played an important role in our
modeling process. We used a few different ways of writing the code,
using some later iterations to condense it in a neater and less unwieldy
form. We also used different iterations of our code to choose the best
ranges for sweeping the initial temperature and k values so that the
model wasn't cluttered with too much data and made sense.

    \hypertarget{abstract}{%
\subsection{Abstract}\label{abstract}}

In this project, we asked how warm should you make your coffee, knowing
that you aren't going to drink it until you get to the AC and that
you'll be in the Nord for an hour beforehand, and instead of making your
coffee hotter, could you change the \emph{k} value to get the same
results?

We found that with an initial temperature of 70 degrees Celsius and a k
value of 0.033, the coffee was too cold to drink after the set time
period. The following graph shows the original loss in temperature
before sweeping any parameters.

    \begin{Verbatim}[commandchars=\\\{\}]
{\color{incolor}In [{\color{incolor}21}]:} \PY{n}{plot}\PY{p}{(}\PY{n}{results}\PY{o}{.}\PY{n}{T}\PY{p}{,} \PY{n}{label}\PY{o}{=}\PY{l+s+s1}{\PYZsq{}}\PY{l+s+s1}{coffee}\PY{l+s+s1}{\PYZsq{}}\PY{p}{)}
         \PY{n}{decorate}\PY{p}{(}\PY{n}{xlabel}\PY{o}{=}\PY{l+s+s1}{\PYZsq{}}\PY{l+s+s1}{Time (minutes)}\PY{l+s+s1}{\PYZsq{}}\PY{p}{,}
                  \PY{n}{ylabel}\PY{o}{=}\PY{l+s+s1}{\PYZsq{}}\PY{l+s+s1}{Temperature (C)}\PY{l+s+s1}{\PYZsq{}}\PY{p}{)}
\end{Verbatim}


    \begin{center}
    \adjustimage{max size={0.9\linewidth}{0.9\paperheight}}{output_47_0.png}
    \end{center}
    { \hspace*{\fill} \\}
    
    When we swept the initial temperature, we found that the highest
possible initial temperatures (100 C) didn't produce final temperatures
that were warm enough. At an initial temperature of 100 C (boiling point
of water) the final temperature is just over 29 C as shown in the graph
below.

    \begin{Verbatim}[commandchars=\\\{\}]
{\color{incolor}In [{\color{incolor}22}]:} \PY{n}{label} \PY{o}{=} \PY{l+s+s1}{\PYZsq{}}\PY{l+s+s1}{k = }\PY{l+s+s1}{\PYZsq{}} \PY{o}{+} \PY{n+nb}{str}\PY{p}{(}\PY{n}{k}\PY{p}{)}
         \PY{n}{plot}\PY{p}{(}\PY{n}{init\PYZus{}Temp}\PY{p}{,} \PY{n}{label}\PY{o}{=}\PY{n}{label}\PY{p}{)}
         
         \PY{n}{decorate}\PY{p}{(}\PY{n}{xlabel}\PY{o}{=}\PY{l+s+s1}{\PYZsq{}}\PY{l+s+s1}{Initial Temperature (degrees Celsius)}\PY{l+s+s1}{\PYZsq{}}\PY{p}{,}
                  \PY{n}{ylabel}\PY{o}{=}\PY{l+s+s1}{\PYZsq{}}\PY{l+s+s1}{Final Temperature}\PY{l+s+s1}{\PYZsq{}}\PY{p}{)}
\end{Verbatim}


    \begin{center}
    \adjustimage{max size={0.9\linewidth}{0.9\paperheight}}{output_49_0.png}
    \end{center}
    { \hspace*{\fill} \\}
    
    Our final parameter sweep, which swept both initial temperature and k
value better answered our question. The following graph indicates our
results.

    \begin{Verbatim}[commandchars=\\\{\}]
{\color{incolor}In [{\color{incolor}23}]:} \PY{k}{for} \PY{n}{k} \PY{o+ow}{in} \PY{n}{k\PYZus{}array}\PY{p}{:}
             \PY{n}{label} \PY{o}{=} \PY{l+s+s1}{\PYZsq{}}\PY{l+s+s1}{k = }\PY{l+s+s1}{\PYZsq{}} \PY{o}{+} \PY{n+nb}{str}\PY{p}{(}\PY{n}{k}\PY{p}{)}
             \PY{n}{plot}\PY{p}{(}\PY{n}{frame}\PY{p}{[}\PY{n}{k}\PY{p}{]}\PY{p}{,} \PY{n}{label}\PY{o}{=}\PY{n}{label}\PY{p}{)}
             
         \PY{n}{decorate}\PY{p}{(}\PY{n}{xlabel}\PY{o}{=}\PY{l+s+s1}{\PYZsq{}}\PY{l+s+s1}{Starting Temp. (C)}\PY{l+s+s1}{\PYZsq{}}\PY{p}{,}
                  \PY{n}{ylabel}\PY{o}{=}\PY{l+s+s1}{\PYZsq{}}\PY{l+s+s1}{Final Temp (C)}\PY{l+s+s1}{\PYZsq{}}\PY{p}{,}
                  \PY{n}{loc}\PY{o}{=}\PY{l+m+mi}{1}\PY{p}{)}
\end{Verbatim}


    \begin{center}
    \adjustimage{max size={0.9\linewidth}{0.9\paperheight}}{output_51_0.png}
    \end{center}
    { \hspace*{\fill} \\}
    
    As shown, to have a final coffee temperature that is warm enough to
drink, a k value of 0.01 or less would be needed along with an initial
temperature of at least 80 degrees Celsius.


    % Add a bibliography block to the postdoc
    
    
    
    \end{document}
